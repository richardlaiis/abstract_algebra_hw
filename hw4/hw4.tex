\documentclass[12pt]{article}
\usepackage[margin=2cm, a4paper]{geometry}
\usepackage{setspace}
\usepackage{float}
\usepackage{amsmath}
\usepackage{fancyvrb}
\usepackage{amssymb}
\usepackage{enumitem}

\usepackage{amsthm}
\usepackage[english]{babel}
\newtheorem{theorem}{Theorem}

\usepackage{xeCJK}
\setCJKmainfont{Noto Serif CJK TC}
\DeclareMathOperator{\lcm}{lcm}
\newcommand{\mytitle}{Abstract algebra I Homework 4}
\newcommand{\myauthor}{B13902022 賴昱錡}

% to fit the hw format
\renewcommand{\thesection}{\arabic{section})}
\renewcommand{\thesubsection}{(\alph{subsection})}

\usepackage{fancyhdr}
\pagestyle{fancy}
\fancyhead{}
\fancyhead[L]{\mytitle}
\fancyhead[R]{\myauthor}

\title{\mytitle}
\author{\textbf{\myauthor}}
\date{}

\begin{document}

\onehalfspacing
\maketitle

% \noindent\rule{\textwidth}{1pt}
\section{}
\subsection{}
Define the homomorphism $\varphi$ as below, where the congruent class modulo $p$ is denoted as $[x]_p$:
$$\varphi:\mathbb{Z}/p\mathbb{Z}\times \mathbb{Z}/q\mathbb{Z}\rightarrow \mathbb{Z}/pq\mathbb{Z}\ \varphi(([x]_p,[x]_q))=[x]_{pq}$$
Suppose $[x]_p=[y]_p,[x]_q=[y]_q$, then $\varphi(([x]_p,[x]_q))=[x]_{pq}$. Since $p|(x-y)$ and $q|(x-y)$, we have $pq|(x-y), [x]_{pq}=[y]_{pq}$. Hence, $[x]_{pq}=[y]_{pq}=\varphi(([y]_{p},[y]_{q}))=\varphi(([x]_p,[x]_q))$. Thus $\varphi$ is well-defined.

$\varphi(([x]_p,[x]_q)+([y]_p,[y]_q))=\varphi(([x+y]_p,[x+y]_q))=[x+y]_{pq}=[x]_{pq}+[y]_{pq}=\varphi(([x]_p,[x]_q))+\varphi(([y]_p,[y]_q))$, so $\varphi$ is a homomorphism.

Suppose $\varphi([x]_p,[x]_q)=0$, $x$ must be multiple of $pq$, hence, $([x]_p,[x]_q)=([0]_p,[0]_q)$. Since $\ker{\varphi}=\{([0]_p,[0]_q)\}$, $\varphi$ is injective, and obviously $|\mathbb{Z}/p\mathbb{Z}\times \mathbb{Z}/q\mathbb{Z}|=|\mathbb{Z}/pq\mathbb{Z}|=pq$, $\varphi$ is also surjective. By proposition above, $\varphi$ is an isomorphism.
\subsection{}
Let $G=\langle g \rangle,H=\langle h\rangle, |G|=n, |H|=m$. If $G\times H$ is cyclic, there exists an integer $d$ such that $(g,h)^d=(e_G,e_H)$. Since $G,H$ are cyclic, we have $n|d,m|d\rightarrow\lcm{(n,m)}|d$. The minimum integer we can choose for $d$ is $\lcm{(n,m)}$, it's also the order of $G\times H$. Since $|G\times H|=nm=\lcm{(n,m)}$, we conclude that $\gcd{(n,m)}=1$.

Suppose $\gcd{(n,m)}=1$, $\langle (g,h)\rangle$ can generate $G\times H$, since the least integer $d$ such that $(g,h)^d=(e_G,e_H)$ is $\lcm{(n,m)}=nm$, which equals to the order of $G\times H$. Thus, $G\times H$ is cyclic if and only if $\gcd{(|G|,|H|)}=1$.
\subsection{}
$S_3=(e,(12),(13),(23),(123),(132))$, the only proper subgroups are 
$$\{e\}, \{e,(12)\}, \{e,(13)\},\{e, (23)\},\{e,(123),(132)\}$$
 Since every two distinct subgroups follow the property: their orders are coprime and both are cyclic, their direct product should also be cyclic group by the result of last subproblem. But $S_3$ is not cyclic, thus it's not direct product of any of its proper subgroups.


\section{}
$G$ is the dicyclic group $\mathrm{Dic}_3$, it contains $\{e_G,a,a^2,a^3,a^4,a^5,ab,a^2b,a^3b,a^4b,a^5b\}$, where $e_G$ is its identity. The order is $12$.

$H$ is the dihedral group $D_3$, it contains $\{e_H, r,r^2,r^3,r^4,r^5,rs,r^2s,r^3s,r^4s,r^5s\}$, where $e_H$ is its identity. The order is $12$.

Consider the order of each element in $H$, $|e_H|=1,|r|=6,|r^2|=3,|r^3|=2,|r^4|=3,|r^5|=6$. Since $sr=r^{-1}s, sr^2=r^{-1}sr=r^{-2}s,\dots\Rightarrow sr^{i}=r^{-i}s$ and $s^2=1$, all elements in the form of $r^is$ have order of $2$. Since $(r^is)^2=r^isr^is=r^ir^{-i}ss=e_H$.

Consider the order of each element in $G$, $|e_G|=1,|a|=6,|a^2|=3,|a^3|=2,|a^4|=3,|a^5|=6$. Since $ba=a^{-1}b, ba^2=a^{-1}ba=a^{-2}b\dots\Rightarrow ba^{i}=a^{-i}b$ and $b^2=a^3,b^4=1$, all the elements in the form of $a^ib$ have order of $4$ ($(a^ib)^2=a^iba^ib=b^2=a^3$, the order of $a^3$ is $2$).

Suppose there exists an isomorphism $\varphi$ from $H$ to $G$, for any element $h\in H$, we have $|\varphi(h)|$ divides $|h|$, since if $|h|=m$, then $\varphi(h^m)=\varphi(e_H)=\varphi(h)^m=e_G$, so the order of $\varphi(h)\in G$ have the order divides $m$. But the elements with order $4$ in $G$ can't divides the orders of any of $H$, hence, $G$ and $H$ can't be isomorphic.

\section{}
\subsection{}
By definition, the orbit of $\langle (12)\rangle=\{e,(12)\}$ is $\{\{1,2\},\{3\},\{4\}\}$, and the orbit of $\langle (123)\rangle=\{e,(123),(132)\}$ is $\{\{1,2,3\},\{4\}\}$. $V=\{e,(12)(34),(13)(24),(14)(23)\}$, its orbit is $\{\{1,2,3,4\}\}$.

\subsection{}
$C_4=\langle (1234)\rangle$. It's a subgroup of $S_4$, since every elements $\sigma^i\in C_4$ for some integer $i$ in $\{0,1,\dots,4\}$, it has a inverse $\sigma^{4-i}$ ($\sigma^0$ is considered as identity) in $C_4$. Also $C_4$ is clearly closed under multiplication. Its orbit is also $\{\{(1234)\}\}$.
\subsection{}
Suppose $\sigma\in S_n$, where $n\ge 3$. If $(12)\sigma=\sigma(12)$, then $1,2$ are either fixed or swapped. If $(23)\sigma=\sigma(23)$, then $3,4$ are also either fixed or swapped. If $\sigma$ commutes with $(12)$ and $(23)$, then $1,2,3$ must be the fixed points of $\sigma$. By simple induction, if $\sigma$ commutes with $(12),(23),(34),\dots,(n-1,n)$, then $1,2,\dots,n$ are all fixed points of $\sigma$. Hence $\sigma$ must be the identity of $S_n$, i.e., $Z(S_n)=\{e\}$. Since $4\ge3$, $Z(S_4)=\{e\}$ is trivial.


\section{}
\subsection{}
Define for each $g\in G$, the map:
$$\varphi_g: S\rightarrow S,\varphi_g(s)=g\cdot s$$. Then $\varphi_g$ is a permutation of $S$ (i.e. a bijection). Indeed, its inverse is $\varphi_{g^{-1}}$ because for every $s\in S$:
$$\varphi_g(\varphi_{g^{-1}}(s))=g\cdot(g^{-1}\cdot s)=s$$
And similarily $\varphi_{g^{-1}}(\varphi_{g}(s))=s$. Thus, $\varphi_g$ is bijective, $\varphi_g\in \mathrm{Perm}(S)$. 

Now define $f:G\rightarrow \mathrm{Perm}(S),\ f(g)=\varphi_g$, since $f(gh)(s)=\varphi_{gh}(s)=(gh)\cdot s$ and $(f(g)\circ f(h))(s)=g\cdot (h\cdot s)=(gh)\cdot s$. Hence, $G\rightarrow \mathrm{Perm}(S)$ is a homomorphism.

\subsection{}
By last subproblem, $G\rightarrow\mathrm{Perm}(S)$ induced a homomorphism $\phi$.
$$\phi:G\rightarrow\mathrm{Perm}(S),\phi(x)(gH)=x\cdot(gH)=(xg)H$$

$x\in\ker{\phi}$ if and only if $\phi(x)(gH)=gH\forall g\in G$, i.e., $x\cdot(gH)=(xg)H=gH$. $(xg)H=gH$ must holds for all $g\in G$. Choose $g=e$, we have $xH=H$, by the property of coset, $x\in H$. Thus, $\ker{\phi}\subseteq H$.

\subsection{}
$|G|/|H|=[G:H]=n$. Let $G$ acts on the set of left cosets of $H$ in $G$, which we will denote as $X=\{gH|g\in G\}$, by 4(a) and 4(b), here induces a homomorphism.
$$\phi:G\rightarrow\mathrm{Perm}(X),\phi(x)(gH)=x\cdot(gH)=(xg)H,gH\in X$$

By the first isomorphism theorem, $\ker{\phi}\triangleleft G$, also by 4(b) $\ker{\phi}\subseteq H$. Since no nontrivial normal subgroup of $G$ is contained in $H$, we concludes that $\ker\phi=\{e\}$.

The first isomorphism theorem states that $G/\ker{\phi}\cong\mathrm{Im}{\phi}$, also $G/\{e\}\cong G$, hence we have $G\cong\mathrm{Im}{\phi}$. Since the image of $\phi$ are some of the permutations on $X$ where $|X|=n$, clearly $G$ is isomorphic to a subgroup of $S_n$.

\end{document}
