% compile this file with xelatex
\documentclass[12pt]{article}
\usepackage{graphicx}
\usepackage[margin=2cm, a4paper]{geometry}
\usepackage{setspace}
\usepackage{pdfpages}
\usepackage{float}
\usepackage{ctex}
\usepackage{amsmath}
\usepackage{fancyvrb}
\usepackage{amssymb}
\usepackage{minted}
\usepackage{enumitem}
\usepackage{csquotes}
\usepackage[colorlinks,linkcolor=blue]{hyperref}

\usepackage{xeCJK}
\setCJKmainfont{Noto Sans TC}

\renewcommand{\contentsname}{Contents}
\renewcommand{\figurename}{Figure}
\renewcommand{\tablename}{Table}
\hypersetup{
    colorlinks=true,
    linkcolor=black,
    filecolor=magenta,      
    urlcolor=blue,
}
\newcommand{\mytitle}{Introduction to Algebra (I) Homework 1}
\newcommand{\myauthor}{B13902022 賴昱錡}

\usepackage{fancyhdr}
\pagestyle{fancy}
\fancyhead{}
\fancyhead[L]{\mytitle}
\fancyhead[R]{\myauthor}

\title{\mytitle}
\author{\textbf{\myauthor}}
\date{\today}

\begin{document}

% \onespacing
\maketitle

\section{}
\subsection*{(a)}
By some calculation as belows, $i=2,3,6$ satisfies the given condition that $f_i(f_i(x))=x$ asides from $i=1$.
\begin{align*}
    f_2(f_2(x))=1-(1-x)=x \\
    f_3(f_3(x))=\frac{1}{\frac{1}{x}}=x \\
    f_6(f_6(x))=\frac{\frac{x}{x-1}}{\frac{x}{x-1}-1}=x
\end{align*}
\subsection*{(b)}
By some observation, $f_1(x)$ satisfies the condition $f_i(x)=x$, $f_2(x),f_3(x),f_6(x)$ satisfies $f_i(f_i(x))=x$, and $f_4(x),f_5(x)$ satisfies the condition $f_i(f_i(f_i(x)))=x$.

For set $S$, we can do the same operation several times to form original permutation. The 1st method is doing $(1)$ for one time, the 2nd method is doing $(12)$, $(23)$ or $(13)$ for 2 times, and the 3rd method is doing $(123)$ or $(132)$ for 3 times.

Thus, considering the counts of composition and the operations, we can correspond $f_2(x)$ to $(12)$, $f_3(x)$ to $(23)$, $f_6(x)$ to $(13)$. Also $f_1(x)$ to $(1)$, and $f_(4)$ to $(123)$, and $f_5(x)$ to $(132)$.
\subsection*{(c)}
For $i,j\in{2,3,6},\ i\ne j$, considering all possible compositions, thus, $f_i(f_j(x))\ne f_j(f_i(x))$ is true.
\begin{align*}
f_2(f_3(x))=1-\frac{1}{x} \\
f_3(f_2(x))=\frac{1}{1-x} \\
f_2(f_6(x))=1-\frac{x}{x-1} \\
f_6(f_2(x))=\frac{1-x}{-x} \\
f_3(f_6(x))=\frac{x-1}{x} \\
f_6(f_3(x))=\frac{\frac{1}{x}}{\frac{1}{x}-1}=\frac{1}{1-x}
\end{align*}
Similarily, consider their corresponding element in $S$, we have: (In my correspondence, if $f_i$ corresponds to operation a, $f_j$ corresponds to operation $b$, then $f_i(f_j(x))$ corresponds to doing $a$ then $b$ ($a\rightarrow b$))
\begin{itemize}
    \item $(12)\rightarrow(13)$ results in $(132)$.
    \item $(13)\rightarrow(12)$ results in $(123)$.
    \item $(12)\rightarrow(23)$ results in $(123)$.
    \item $(23)\rightarrow(12)$ results in $(132)$.
    \item $(13)\rightarrow(23)$ results in $(132)$.
    \item $(23)\rightarrow(13)$ results in $(123)$.
\end{itemize}
The corresponding element in $S$ also satisfies the given condition.
\subsection*{(d)}
By 1(b), $f_1(x)$ is corresponding to $(1)$, and $f_4(x)$ to $(123)$, and $f_5(x)$ to $(132)$.
\section{}
\subsection*{(a)}
The elements satisfying the condition when $n=5$ include $1,2,3$. For $x=1$, we can choose $y=1$. For $x=2$, we can choose $y=3$. For $x=3$, we can choose $y=2$. For $x=4$, we can choose $y=4$. There are 4 elements satisfying the condition.
\subsection*{(b)}
For $n=6$:
\begin{itemize}
    \item $x=1,y=1\Rightarrow1*1\equiv 1\pmod6$
    \item $x=5,y=5\Rightarrow5*5\equiv 1\pmod6$
\end{itemize}
\subsection*{(c)}
For $n=8$:
\begin{itemize}
    \item $x=1,y=1\Rightarrow1\equiv 1\pmod8$
    \item $x=3,y=3\Rightarrow9\equiv 1\pmod8$
    \item $x=5,y=5\Rightarrow25\equiv 1\pmod8$
    \item $x=7,y=7\Rightarrow49\equiv 1\pmod8$
\end{itemize}
\subsection*{(d)}
For $n=13$:
\begin{itemize}
    \item $x=1,y=1\Rightarrow1\equiv 1\pmod{13}$
    \item $x=2,y=7\Rightarrow14\equiv 1\pmod{13}$ 
    \item $x=3,y=9\Rightarrow27\equiv 1\pmod{13}$
    \item $x=4,y=10\Rightarrow40\equiv 1\pmod{13}$
    \item $x=5,y=8\Rightarrow40\equiv 1\pmod{13}$
    \item $x=6,y=11\Rightarrow66\equiv 1\pmod{13}$
    \item $x=7,y=2\Rightarrow14\equiv 1\pmod{13}$
    \item $x=8,y=5\Rightarrow40\equiv 1\pmod{13}$
    \item $x=9,y=3\Rightarrow27\equiv 1\pmod{13}$
    \item $x=10,y=4\Rightarrow40\equiv 1\pmod{13}$
    \item $x=11,y=6\Rightarrow66\equiv 1\pmod{13}$
    \item $x=12,y=12\Rightarrow144\equiv 1\pmod{13}$
\end{itemize}
\subsection*{(e)}
For $n=30$:
\begin{itemize}
    \item $x=1,y=1\Rightarrow1\equiv 1\pmod{30}$
    \item $x=7,y=13\Rightarrow91\equiv 1\pmod{30}$
    \item $x=11,y=11\Rightarrow121\equiv 1\pmod{30}$
    \item $x=13,y=7\Rightarrow91\equiv 1\pmod{30}$
    \item $x=17,y=23\Rightarrow391\equiv 1\pmod{30}$
    \item $x=19,y=19\Rightarrow361\equiv 1\pmod{30}$
    \item $x=23,y=17\Rightarrow391\equiv 1\pmod{30}$
    \item $x=29,y=29\Rightarrow841\equiv 1\pmod{30}$
\end{itemize}
\section{}
\subsection*{(a)}
We have to prove that "There exists integer $a,b$ such that $ax+bn=1$." is \textbf{necessary and sufficient} for $x\in\mathbb{Z}/n\mathbb{Z}^{\times}$ (Here $x\in\mathbb{Z}/n\mathbb{Z}$).

For any $x\in(\mathbb{Z}/n\mathbb{Z})^{\times}$, by definition, there exists integer $a\in[1,n-1]$ (Because $a\in\mathbb{Z}/n\mathbb{Z}$, the condition $a=0$ is impossible for $ax\equiv 1\pmod{n}$), such that $ax\equiv 1\pmod{n}$, so $ax$ can be written as the form $ax=-bn+1,b\in\mathbb{Z},b=\frac{ax-1}{-n}$. Thus, there exists integer $a,b$ such that $ax+bn=1$, and the \textbf{necessity} of "There exists integer $a,b$ such that $ax+bn=1$." has been proven. 

If there exists integer $a,b$ such that $ax+bn=1$, integer $a$ can be written as the form $a=un+v, u\in\mathbb{Z},v\in\mathbb{Z}/n\mathbb{Z}$ (By basic division). Make a substitution:
\begin{align*}
    (un+v)x+bn=1 \\
    vx=-unx-bn+1=n(-ux-b)+1
\end{align*}
We can observe that $vx\equiv 1\pmod{n}$, thus, $x\in(\mathbb{Z}/n\mathbb{Z})^{\times}$, and the sufficiency has been proven.

In conclusion, "There exists integer $a,b$ such that $ax+bn=1$." is \textbf{necessary and sufficient} for $x\in\mathbb{Z}/n\mathbb{Z}^{\times}$ (Here $x\in\mathbb{Z}/n\mathbb{Z}$).
\subsection*{(b)}
In (b), we need to prove that "$n$ is prime" is \textbf{necessary and sufficient} for $(\mathbb{Z}/n\mathbb{Z})^{\times}$ having $n-1$ elements.

Before proving the sufficiency, we have to prove the theorem called \textbf{Bézout's Identity}:

Also, for simplicity, $(a,b)$ means $\gcd{(a,b)}$ in the following proof.

\begin{displayquote}
Let $a,b\in\mathbb{Z},ab\ne0$

$d=\gcd{(a,b)}$ be the greatest common divisor of $a$ and $b$. 

Then $\exists x,y\in\mathbb{Z}$ such that $ax+by=d$. Also, $d$ is the smallest positive integer combination of $a$ and $b$.
\end{displayquote}

\noindent\textbf{Proof:}

Given any two non-zero integer $a,b$, Let set $S=\{ax+by:x,y\in\mathbb{Z}\land ax+by>0\}$

It's trivial that $S$ is not an empty set (For example, $a>0,x=1,y=0$ or $a<0,x=1,y=0$, $ax+by\in S$, thus, $S$ is not an empty set). Since all elements in $S$ are positive integers, by well ordering principle, $S$ contains a least element $d$. And write it as the form $d=au+bv$, where $u$ and $v$ are integers.

Consider $a$'s euclidean division: $a=qd+r, q\in\mathbb{Z}, 0\le r < d$, we have:
\[r=a-qd=a-q(au+bv)=a(1-qu)-bqv\]
Because both $1-qu$ and $qv$ are integers, $r\in S\cup \{0\}$ (because $0\le r < d$). Also, $d$ is the least element in $S$, this implies that $r$ is not belonging to $S$, it must be $0$. Thus, $d|a$. Similarily, $d|b$.

Consider arbitrary common divisor $c$ of $a,b$, $\exists s,t$ such that $a=cs,b=ct$. So, $d=au+bv=c(us+vt)$, because $us+vt\in \mathbb{Z}$, we know $c|d\land c\le d$.

Since $d$ is greater than all divisors, $d=\gcd{(a,b)}$, it's also the least element in $S$ by previous definition.

\noindent\rule{\textwidth}{1pt}

To prove the sufficiency of $n$ being prime, assume $n$ is prime. Then for every $x\in(\mathbb{Z}/n\mathbb{Z}), x\ne 0$ we have $(x,n)=1$. By Bézout's Identity, there exists integers $a,b$ such that $ax+bn=1$. This implies $x\in(\mathbb{Z}/n\mathbb{Z})^{\times}$ by 3(a). Thus, there are $n-1$ elements in $(\mathbb{Z}/n\mathbb{Z})^{\times}$ (Since except for $0$, $(\mathbb{Z}/n\mathbb{Z})$ contains $n-1$ elements).

To prove the necessity of $n$ being prime, assume there are $n-1$ elements in $(\mathbb{Z}/n\mathbb{Z})^{\times}$, the group has all integers on the interval $[1,n-1]$. And this implies that there exists integer $a$ and $b$ such that $ax+bn=1$ by 3(a). (choose arbitrary $x\in(\mathbb{Z}/n\mathbb{Z})^{\times}$) 

Let $\gcd{(x,n)}=k\ne n, x=ku, n=kv$, $u,v\in \mathbb{Z}$. Pluggin it in $ax+bn=1$ we get $k(ua+vb)=1$, since $ua+vb\in \mathbb{Z}$, $k\ne 1$ is impossible. So, $(x,n)=1$. Because all integers on $[1,n-1]$ is coprime to $n$, $n$ is a prime number obviously. Here the necessity is proved.

In conclusion, "$n$ is prime" is \textbf{necessary and sufficient} for $(\mathbb{Z}/n\mathbb{Z})^{\times}$ having $n-1$ elements. Thus, $(\mathbb{Z}/n\mathbb{Z})^{\times}$ has $n-1$ elements if and only if $n$ is prime.
\section{}
\subsection*{(a)}
Suppose $e$ and $e'$ are identity elements of group $(G,*)$, by the definition of group, $\forall g\in G$, there are:
$$g*e=e*g=g,\ g*e'=e'*g=g$$
Thus, $e=e*e'=e'*e=e'$, the identity element of a group is unique.

Suppose $h$ and $h'$ are the inverse element of an element $x\in G$, and $e$ is the identity element of group $(G,*)$, by the definition:
$$x*h=h*x=e,\ x*h'=h'*x=e$$
And $h*(x*h')=(h*x)*h'=e*h'=h'$, thus, the inverse for every $x\in G$ is unique.
\subsection*{(b)}
We can consider four cases: a group having 1,2,3 and 4 elements (obviously an empty set is not a group due to the lack of identity and inverse).

Any group with only one element (Let it be $\{x\}$) trivially follows the commutative rule, since the order doesn't matter:
$$x*x=x\Rightarrow x=x^{-1}=e$$

A group $G$ with 2 elements must be in the form $\{e,a\}$ (By definition, group must has one unique identity element), where $e$ is the identity and $a$ is a  non-identity element. (In the case $a^{-1}=a, e^{-1}=e$). By the definition, $a*e=e*a=a$. Thus, every group with 2 elements is abelian.

Similarily, a group $G$ with 3 elements must be in the form $\{e, a, b\}$ ($a\ne b\ne e$, $e$ is the identity), since $a\ne e,b\ne e,ab\in G$, we have:
$$a*b=e$$

$a$ and $b$ are inverse to each other, $a*b=b*a$. This is the only non-trival case, $a*e=e*a$ and $b*e=e*b$ are true by definition. Thus, every group with 3 elements is also abelian group.


Similarily, a group $G$ with 4 elements must be in the form $\{e, a, b, c\}$ ($a\ne b\ne c\ne e$, $e$ is the identity). Assume that the group is \textbf{not} an abelian group, that is, there exists one pair of non-identity elements (Without loss of generosity, let the pair be $a$ and $b$) such that $a*b\ne b*a$.

$a*b\ne a,a*b\ne b$ since the identity is unique. Also, $a*b\ne e$, because $a*b=e$ implies $b*a=e$ as well. To make $a*b\in G$, the only possibility is $a*b=c$.

Consider $b*a$, it is not equal to $a$, $b$ and $e$ because of the same reason mentioned in last paragraph. But $b*a\in G$ by the definition of group, the only possibility is $b*a=a*b=c$, which creates a contradiction. Thus, any group with 4 elements are abelian.

After proving as above, we can have the conclusion that if $G$ has at most four elements, for all $x,y\in G$, we have $x*y=y*x$.
\subsection*{(c)}
If every element $x\in G$ satisfies $x*x=e$, then $x=x^{-1}$. So:
\begin{align*}
    x*y &= x^{-1}*y^{-1} \\
    &= (y*x)^{-1} \\
    &= y*x
\end{align*}
Obviously, for all $x,y\in G$ we have $x*y=y*x$.
\end{document}
% how to display codes?
% \begin{minted}[frame=lines,framesep=2mm,baselinestretch=1.2,linenos,breaklines]{python}
% \end{minted}

% how to display images?
% \begin{figure}[H]
%     \centering
%     \includegraphics[width=0.5\linewidth]{}
%     \caption{Caption}
% \end{figure}
% test