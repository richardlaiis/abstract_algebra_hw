\documentclass[12pt]{article}
\usepackage[margin=2cm, a4paper]{geometry}
\usepackage{setspace}
\usepackage{float}
\usepackage{amsmath}
\usepackage{fancyvrb}
\usepackage{amssymb}
\usepackage{enumitem}
\usepackage{amsthm}
\usepackage[english]{babel}

\usepackage{xeCJK}
\setCJKmainfont{Noto Sans TC}

\newcommand{\mytitle}{Abstract algebra I Homework 2}
\newcommand{\myauthor}{B13902022 賴昱錡}

% to fit the hw format
\renewcommand{\thesection}{\arabic{section})}
\renewcommand{\thesubsection}{(\alph{subsection})}

\usepackage{fancyhdr}
\pagestyle{fancy}
\fancyhead{}
\fancyhead[L]{\mytitle}
\fancyhead[R]{\myauthor}

\title{\mytitle}
\author{\textbf{\myauthor}}
\date{Due: 24th September 2025}

\begin{document}

\onehalfspacing
\maketitle

\section{} % 1
\subsection{} %a
Take the sum of $14$ and $13$, and it's $27$ modulo $30$, $27\notin G_1$ thus, $G_1$ is not a subgroup of $G$.
\subsection{} %b
We can check some necessary properties a subgroup must follow:
\begin{itemize}
    \item All elements of $G_2$ are also in $G$, $G_2\in G$
    \item $\exists e$ such that $\forall g\in G_2,g+e=e+g=g$. There $e=0$.
    \item $0+0\equiv 0\pmod{30}$, thus, $g^{-1}=0$ when $g=0$. All the other non-zero elements $\in G_2$ can be written as the form $2k,k\in[1,14],k\in\mathbb{N}$. Assume $g=2k$, then there must exists an element $h=2(15-k)\in G_2$ such that $g+k\equiv 0\pmod{30}$. Thus, every element in $G_2$ has an inverse element.
\end{itemize}
\subsection{} %c
Take the sum of $1$ and $29$, and it's $0$ modulo $30$, $0\notin G_3$ thus, $G_3$ is not a subgroup of $G$.





\newpage
\section{} % 2
\subsection*{(i)}
\begin{proof}
Since $H$ is not empty, we can choose $x,y\in G$.

By the closedness of inverse, the inverse of $x$ exists and belongs to the $H$, let $y=x^{-1}$.

By the closedness of $*$, $x*x^{-1}=e\in H$, where $e$ is the identity element of $H$. Thus, the identity of $H$ exists.

Since $H$ is closed under products, and inverse for each element exists, and the identity for $H$ exists. It's a group and $H\subset G$, so $H$ is a subgroup of $G$.
\end{proof}
\subsection*{(ii)}
For simplicity, I denote the determinant of a n by n matrix $A$ as $|A|$.

Since the determinant of an identity matrix $I_n$ is $1$, $SL_n(\mathbb{R})\ne \emptyset$.

For any matrices $a,b\in SL_n(\mathbb{R})$, suppose $c=ab$, then $c$ must be a real matrix (all entries are real), also, $|c|=|a||b|=1*1=1$, the determinant of $c$ is also $1$. Thus, $c\in SL_n(\mathbb{R})$. Here proves the closedness of matrix multiplication.

\noindent\rule{\textwidth}{1pt}
\noindent \textbf{Claim 1: } Real $n\times n$ matrix $A$ is invertible if and only if $|A|\ne0$
\begin{proof}
Suppose $A$ is invertible, then there exists a matrix $B$ such that $AB=I$. $|I|=|A||B|=1$, $|A|$ can't be zero. 

Assume $|A|\ne 0$, then $B=\frac{1}{|A|}\mathrm{adj}(A)$ (B is also a real $n\times n$ matrix) satisfies $AB=BA=I$ where $\mathrm{adj}(A)$ is the classical adjoint matrix of A and I is the identity matrix.

Thus, $|A|\ne 0$ is necessary and sufficient.
\end{proof}
\noindent\rule{\textwidth}{1pt}

By claim 1, every element in $H$ has its inverse due to their non-zero determinant. Suppose A is any matrix in $H$, and its inverse is $A^{-1}$, then $AA^{-1}=A^{-1}A=I,|A||A^{-1}|=|I|=1$, thus, $|A^{-1}|=1$.

Hence, the inverse of A, i.e., $A^{-1}$ is also in $H$. Here the closedness of inverse is proved. By the subgroup criterion proved in 2(i), $\mathrm{SL}_n{\mathbb{R}}$ is a subgroup of $\mathrm{GL}_n{\mathbb{R}}$.


\newpage
\section{} % 3
\subsection{} %a
\begin{proof}
Let's call the two sets $A$ and $B$. $A=\{1,2,\dots,n\},B=\{1,2,\dots,n\}$. And $A$ is mapped to $B$.

Since the map is bijective, for $1$ in $A$, there are $n$ choices to be mapped, after $1$ is mapped, $2$ in $A$ has $n-1$ choices to be mapped, and so on.

Thus, there are $n(n-1)(n-1)\dots1=n!$ types of bijection, i.e., the order of $S_n$ is $n!$.
\end{proof}
\subsection{} %b
\subsection{} %c
The identity element in the group for matrices multiplication is the identity matrix $I_{2\times2}$.

For $a$, $a=\big(\begin{smallmatrix}0 & -1\\1 & 0\end{smallmatrix}\big)$,$a^2=\big(\begin{smallmatrix}-1 & 0\\0 & -1\end{smallmatrix}\big)$,$a^3=\big(\begin{smallmatrix}0 & 1\\-1 & 0\end{smallmatrix}\big)$,$a^1=\big(\begin{smallmatrix}1 & 0\\0 & 1\end{smallmatrix}\big)=I_{2\times2}$. Thus, $o(a)=4$. For $b$, $b=\big(\begin{smallmatrix}0 & 1\\-1 & -1\end{smallmatrix}\big)$, $b^2=\big(\begin{smallmatrix}-1 & -1\\1 & 0\end{smallmatrix}\big)$,$b^3=\big(\begin{smallmatrix}1 & 0\\0 & 1\end{smallmatrix}\big)=I_{2\times2}$, thus, $o(b)=3$.










\newpage
\section{} % 4
\subsection{} %a
\subsection{} %b
\subsection{} %c












\newpage
\section{} % 5
\subsection{} %a
\begin{proof}
    
\end{proof}
\subsection{} %b
\subsection{} %c
Since elements in the abelian group $(G,*)$ are commutative, i.e., for any $a,b\in G$, we have $a*b=b*a$.

Let's choose one arbitrariry elements $g$, consider the subgroup as $B=\{a_1,a_2,\dots,a_m\}$. Then $gB=\{g*a_1,g*a_2,\dots,g*a_m\}$, and $Bg=\{a_1*g,a_2*g,\dots,a_m*g\}$. since $g*a_i=a_i*g$ for all $i$, we have $gB=Bg$.

Hence, by the definition, every subgroup of an abelian group is normal.
\subsection{} %d


\end{document}
