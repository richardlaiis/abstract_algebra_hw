\documentclass[12pt]{article}
\usepackage[margin=2cm, a4paper]{geometry}
\usepackage{setspace}
\usepackage{float}
\usepackage{amsmath}
\usepackage{fancyvrb}
\usepackage{amssymb}
\usepackage{enumitem}

\usepackage{amsthm}
\usepackage[english]{babel}
\newtheorem{theorem}{Theorem}

\usepackage{xeCJK}
\setCJKmainfont{Noto Sans TC}

\newcommand{\mytitle}{Abstract algebra I Homework 2}
\newcommand{\myauthor}{B13902022 賴昱錡}

% to fit the hw format
\renewcommand{\thesection}{\arabic{section})}
\renewcommand{\thesubsection}{(\alph{subsection})}

\usepackage{fancyhdr}
\pagestyle{fancy}
\fancyhead{}
\fancyhead[L]{\mytitle}
\fancyhead[R]{\myauthor}

\title{\mytitle}
\author{\textbf{\myauthor}}
\date{Due: 24th September 2025}

\begin{document}

\onehalfspacing
\maketitle

\section{} % 1
\subsection{} %a
Take the sum of $14$ and $13$, and it's $27$ modulo $30$, $27\notin G_1$ thus, $G_1$ is not a subgroup of $G$.
\subsection{} %b
We can show that the subset $G_2$ is a subgroup of $G$ by showing $G_2$ is non-empty and closed under addition ($+$) and inverses:
\begin{itemize}
    \item The identity element of $G_2$ is 0, since the result of $g+0=0+g=g$ modulo $30$ is unchanged.
    \item $0+0\equiv 0\pmod{30}$, thus, $g^{-1}=0$ when $g=0$. All the other non-zero elements $\in G_2$ can be written as the form $2k,k\in[1,14],k\in\mathbb{N}$. Assume $g=2k$, then there must exists an element $h=2(15-k)\in G_2$ such that $g+k\equiv 0\pmod{30}$. Thus, every element in $G_2$ has an inverse element in $G_2$.
    \item Take the sum of any two elements $x,y\in G_2$. Their sum is even, and the result modulo 30 must be a even number less than 30, which implies $x+y\in G_2$. Thus, $G_2$ is closed under addition.
\end{itemize}
\subsection{} %c
Take the sum of $1$ and $29$, and it's $0$ modulo $30$, $0\notin G_3$ thus, $G_3$ is not a subgroup of $G$.





\newpage
\section{} % 2
\subsection*{(i)}
\begin{proof}
Since $H$ is not empty, we can choose $x,y\in G$.

By the closedness of inverse, the inverse of $x$ exists and belongs to the $H$, let $y=x^{-1}$.

By the closedness of $*$, $x*x^{-1}=e\in H$, where $e$ is the identity element of $H$. Thus, the identity of $H$ exists.

Since $H$ is closed under products, and inverse for each element exists, and the identity for $H$ exists. It's a group and $H\subset G$, so $H$ is a subgroup of $G$.
\end{proof}
\subsection*{(ii)}
For simplicity, I denote the determinant of a n by n matrix $A$ as $|A|$.

Since the determinant of an identity matrix $I_n$ is $1$, $SL_n(\mathbb{R})\ne \emptyset$.

For any matrices $a,b\in SL_n(\mathbb{R})$, suppose $c=ab$, then $c$ must be a real matrix (all entries are real), also, $|c|=|a||b|=1*1=1$, the determinant of $c$ is also $1$. Thus, $c\in SL_n(\mathbb{R})$. Here proves the closedness of matrix multiplication.

\noindent\rule{\textwidth}{1pt}
\noindent \textbf{Claim 1: } Real $n\times n$ matrix $A$ is invertible if and only if $|A|\ne0$
\begin{proof}
Suppose $A$ is invertible, then there exists a matrix $B$ such that $AB=I$. $|I|=|A||B|=1$, $|A|$ can't be zero. 

Assume $|A|\ne 0$, then $B=\frac{1}{|A|}\mathrm{adj}(A)$ (B is also a real $n\times n$ matrix) satisfies $AB=BA=I$ where $\mathrm{adj}(A)$ is the classical adjoint matrix of A and I is the identity matrix.

Thus, $|A|\ne 0$ is necessary and sufficient.
\end{proof}
\noindent\rule{\textwidth}{1pt}

By claim 1, every element in $SL_n(\mathbb{R})$ has its inverse due to their non-zero determinant. Suppose A is any matrix in $SL_n(\mathbb{R})$, and its inverse is $A^{-1}$, then $AA^{-1}=A^{-1}A=I,|A||A^{-1}|=|I|=1$, thus, $|A^{-1}|=1$.

Hence, the inverse of A, i.e., $A^{-1}$ is also in $SL_n(\mathbb{R})$. Here the closedness of inverse is proved. By the subgroup criterion proved in 2(i), $\mathrm{SL}_n({\mathbb{R}})$ is a subgroup of $\mathrm{GL}_n({\mathbb{R}})$.


\newpage
\section{} % 3
\subsection{} %a
First, we need to prove that the set $S_n$ has $n!$ elements.
\begin{proof}
Let's call the two sets $A$ and $B$. $A=\{1,2,\dots,n\},B=\{1,2,\dots,n\}$. And $A$ is mapped to $B$.

Since the map is bijective, for $1$ in $A$, there are $n$ choices to be mapped, after $1$ is mapped, $2$ in $A$ has $n-1$ choices to be mapped, and so on.

Thus, there are $n(n-1)(n-1)\dots1=n!$ types of bijection.
% I still need to show that S is a group!!!
\end{proof}
Then we need to show that $S_n$ is a group (there the operation is the composition of two bijective functions)
\begin{proof}
For the closedness, if $f,g\in S_n$, then their composition (written as $(f\circ g)(x)=f(g(x))$) is also bijective trivially. Thus, $f\circ g\in S_n$.

For the associativity, suppose we have three bijections $f,g,h\in S_n$, take any integer $x\in{1,2,\dots,n}$. Then $f\circ (g\circ h)(x)=f\circ g(h(x))=f(g(h(x)))$ and $(f\circ g)\circ h(x)=(f\circ g)(h(x))=f(g(h(x)))$. The composition of bijections are associative.

The identity element in $S_n$ is the mapping $\mathrm{id}:\{1,2,\dots,n\}\rightarrow\{1,2,\dots,n\}$ such that for every element $\tau$ in $S_n$, we have $\mathrm{id}\circ\tau(x)=\tau\circ\mathrm{id}(x)$, where $x\in\{1,2,\dots,n\}$.

Since every element in $S_n$ is a bijective mapping, by definition, we must can find an inverse operation/element for every element in $S_n$.
\end{proof}
In conclusion, $S_n$ is a group with order $n!$.
\subsection{} %b
Let's call the subset $H$ ($H\subset S_2$). Obviously the $H$ is not empty. For compositions of any $\tau,\sigma\in H$, we have a bijection fixing $1$ again, thus, $H$ is closed under the composition.

Since every bijection has its inverse operation, also, $\forall \sigma\in H$, $1$ is fixed (always mapped to $1$), $\sigma^{-1}$ is also a bijection fixing 1, which implies $\sigma^{-1}\in H$. Thus, $H$ is closed under taking inverses.

Hence, by the subgroup criterion proved in 2(i), the given subset is a subgroup of $S_4$.

Since the subset is the collection of bijections fixing $1$, the bijective mapping $\{2,3,4\}\rightarrow\{{2,3,4}\}$ have $3!$ possibilities by the proposition in 3(a). Hence, the order of the group is $6$.
\subsection{} %c
The identity element in the group for matrices multiplication is the identity matrix $I_{2\times2}$.

For $a$, $a=\big(\begin{smallmatrix}0 & -1\\1 & 0\end{smallmatrix}\big)$,$a^2=\big(\begin{smallmatrix}-1 & 0\\0 & -1\end{smallmatrix}\big)$,$a^3=\big(\begin{smallmatrix}0 & 1\\-1 & 0\end{smallmatrix}\big)$,$a^4=\big(\begin{smallmatrix}1 & 0\\0 & 1\end{smallmatrix}\big)=I_{2\times2}$. Thus, $o(a)=4$. For $b$, $b=\big(\begin{smallmatrix}0 & 1\\-1 & -1\end{smallmatrix}\big)$, $b^2=\big(\begin{smallmatrix}-1 & -1\\1 & 0\end{smallmatrix}\big)$,$b^3=\big(\begin{smallmatrix}1 & 0\\0 & 1\end{smallmatrix}\big)=I_{2\times2}$, thus, $o(b)=3$.










\newpage
\section{} % 4
\subsection{} %a
\begin{theorem} \textbf{Bézout's identity}

Let $a,b\in\mathbb{Z},ab\ne0$

$d=\gcd{(a,b)}$ be the greatest common divisor of $a$ and $b$. 

Then $\exists x,y\in\mathbb{Z}$ such that $ax+by=d$. Also, $d$ is the smallest positive integer combination of $a$ and $b$.
\end{theorem}
\begin{proof}

Given any two non-zero integer $a,b$, Let set $S=\{ax+by:x,y\in\mathbb{Z}\land ax+by>0\}$

It's trivial that $S$ is not an empty set (For example, $a>0,x=1,y=0$ or $a<0,x=1,y=0$, $ax+by\in S$, thus, $S$ is not an empty set). Since all elements in $S$ are positive integers, by well ordering principle, $S$ contains a least element $d$. And write it as the form $d=au+bv$, where $u$ and $v$ are integers.

Consider $a$'s euclidean division: $a=qd+r, q\in\mathbb{Z}, 0\le r < d$, we have:
\[r=a-qd=a-q(au+bv)=a(1-qu)-bqv\]
Because both $1-qu$ and $qv$ are integers, $r\in S\cup \{0\}$ (because $0\le r < d$). Also, $d$ is the least element in $S$, this implies that $r$ is not belonging to $S$, it must be $0$. Thus, $d|a$. Similarily, $d|b$.

Consider arbitrary common divisor $c$ of $a,b$, $\exists s,t$ such that $a=cs,b=ct$. So, $d=au+bv=c(us+vt)$, because $us+vt\in \mathbb{Z}$, we know $c|d\land c\le d$.

Since $d$ is greater than all divisors, $d=\gcd{(a,b)}$, it's also the least element in $S$ by previous definition.
\end{proof}

\noindent\rule{\textwidth}{1pt}
For simplicity, for a generator $c$ of a cyclic group, the order of it I may type it as $|x|$ or $o(x)$, so are the order of groups.

\noindent \textbf{Claim 2: } If x is the generator of cyclic group $H$, then the order of $H$ is the same as $x$ (If one side of this equality is infinite, so is the other).
\begin{proof}
Let $|x|=n$ and first consider the case when $n<\infty$. The elements $1,x,x^2,\dots,x^{n-1}$ are distinct since if $x^a=x^b,0\le a<b<n$ then $x^{b-a}=1$, which contradict $n$ being the smallest positive power give the identity. Also, we can write any integer power $t$ as the form $t=ns+r,0\le r<n$. Hence, $x^t=x^{ns+r}=(x^n)^sx^r=x^r\in\{1,x,\dots,x^{n-1}\}$, $x$ can generate all elements in $H$.

Suppose $|x|=\infty$ so no power of $x$ is the identity, If $x^a=x^b$ for some $a$ and $b$, with $a<b$, then $x^{b-a}=1$ induced a contradiction. Distinct power of $x$ are distinct elements of $|H|$, so $|H|=\infty$ is true.
\end{proof}

\noindent \textbf{Claim 3: } Let G be an arbitrary group, $x\in G$ and let $m,n\i,n\mathbb{Z}$. If $x^n = 1$ and $x^m = 1$, then $x^d = 1$, where $d=(m,n)$ In particular, if $x^m = 1$ for some $m\in\mathbb{Z}$, then $|x|$ divides $m$.
\begin{proof}
By Theorem 1 there exists integers $a$ and $b$ such that $d=an+bm,d=(m,n)$. Thus, $x^d=x^{an+bm}=(x^n)^a(x^m)^b=1$, this proves the first assertion.

If $x^m=1$, let $n=|x|$. If $m=0$, $n|m$ is trivially true. Assume $m$ is not zero, by preceeding result, $x^d=1,d=(m,n)$. Since $0< d\le n$ and $n$ is the smallest positive power of $x$ which gives the identity, we must have $d=n$, that is $n|m$ as the claim said.
\end{proof}
\noindent \textbf{Claim 4:} Let $G$ be a group, let $x\in G$ and let $a\in\mathbb{Z}-\{0\}$, if $o(x)=n<\infty$, then $o(x^a)=\frac{n}{(n,a)}$. 
\begin{proof}
Let $y=x^a,(n,a)=d$ and write $n=db,a=dc$ for suitable $b,c\in\mathbb{Z},b>0$. Since $d$ is the greatest common divisor, $b$ and $c$ are coprime, i.e., $(b,c)=1$

Note that $y^b=x^{ab}=x^{bdc}=(x^n)^c=1$. By Claim 3 applied to $<y>$, we have $|y||b$, Let $k=|y|$. Then $x^{ak}=y^k=1$. 

By Claim 3 applied to $<x>$, $n|ak$,i.e.,$db|dck$, thus $b|ck$. Since $(b,c)=1$, $b|k$. Since $b$ and $|y|$ divides each other, we have $|y|=o(y)=b$,i.e, $o(x^a)=\frac{n}{(n,a)}$.
\end{proof}
\noindent \textbf{Claim 5:} Let $H=<x>$. Assume $o(x)=n<\infty$. Then $H=<x^a>$ if and only if $(a,n)=1$.
\begin{proof}
If $|x|=n<\infty$. Claim 2 says $x^a$ generates a subgroup of $H$ of order $|x^a|$. The subgroup equasl $|H|$ only when $|x|=|x^a|$. By Claim 4, $|x^a|=|x|$ if and only if $\frac{n}{(a,n)}=n$, i.e., $(n,a)=1$.
\end{proof}
\noindent\rule{\textwidth}{1pt}

What we wants to prove is $(k,n)=1$ is suffcient and necessary for $g^k$ being a generator of $G$.

For the sufficiency of $(k,n)=1$, by Theorem 1, there must exists $a,b\in\mathbb{Z}$ such that $an+bk=1$. Since, $an+bk=1$, $bk=-an+1$ and $(bt)k=-(at)n+t$. We have $g^{(bt)k}=g^{-(at)n}g^t$, and $(g^k)^{bt}=g^t$. For integer $t\in[1,n]$, $g^k$ can generate the group $\{1,g,g^2,\dots,g^{n-1}\}$.

Claim 5 already proves the necessity. Hence, if $g$ is a generator of $G$, then $g^k$ is a generator of $G$ iff $(n,k)=1$.
\subsection{} %b
\subsection{} %c












\newpage
\section{} % 5
\subsection{} %a
\begin{proof}
    
\end{proof}
\subsection{} %b
\subsection{} %c
Since elements in the abelian group $(G,*)$ are commutative, i.e., for any $a,b\in G$, we have $a*b=b*a$.

Let's choose one arbitrariry elements $g$, consider the subgroup as $B=\{a_1,a_2,\dots,a_m\}$. Then $gB=\{g*a_1,g*a_2,\dots,g*a_m\}$, and $Bg=\{a_1*g,a_2*g,\dots,a_m*g\}$. since $g*a_i=a_i*g$ for all $i$, we have $gB=Bg$.

Hence, by the definition, every subgroup of an abelian group is normal.
\subsection{} %d


\end{document}
