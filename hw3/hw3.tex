\documentclass[12pt]{article}
\usepackage[margin=2cm, a4paper]{geometry}
\usepackage{setspace}
\usepackage{float}
\usepackage{amsmath}
\usepackage{fancyvrb}
\usepackage{amssymb}
\usepackage{enumitem}

\usepackage{amsthm}
\usepackage[english]{babel}
\newtheorem{theorem}{Theorem}

\usepackage{xeCJK}
\setCJKmainfont{Noto Serif CJK TC}

\newcommand{\mytitle}{Abstract algebra I Homework 3}
\newcommand{\myauthor}{B13902022 賴昱錡}

% to fit the hw format
\renewcommand{\thesection}{\arabic{section})}
\renewcommand{\thesubsection}{(\alph{subsection})}

\usepackage{fancyhdr}
\pagestyle{fancy}
\fancyhead{}
\fancyhead[L]{\mytitle}
\fancyhead[R]{\myauthor}

\title{\textbf{\mytitle}}
\author{\textbf{\myauthor}}
\date{Due: 1st October 2025}

\begin{document}

\onehalfspacing
\maketitle

\section{} %1
\subsection{} % 1a
\noindent\rule{\textwidth}{1pt}

\textbf{Claim: }If $G$ is cyclic and $H$ is isomorphic to $G$, then $H$ is cyclic.
\begin{proof}
Suppose the isomorphism is $\varphi: G\rightarrow H$ and $G=\langle x\rangle$, then every element of $H$ can be written as the form $\varphi(x^d)$ for some integer $d$, also: $\varphi(x^d)=\varphi(x)^d$. Hence, $H$ is generated by $\varphi(x)$, i.e., by claim above, $H=\langle \varphi(x)\rangle$ is cyclic.
\end{proof}

\noindent\rule{\textwidth}{1pt}


$(\mathbb{Z}/15\mathbb{Z})^{\times}=\{\bar{1},\bar{2},\bar{4},\bar{7},\bar{8},\bar{11},\bar{13},\bar{14}\}$ and $\mathbb{Z}/8\mathbb{Z}=\{\bar{0},\bar{1},\bar{2},\bar{3},\bar{4},\bar{5},\bar{6},\bar{7}\}$. Since $\mathbb{Z}/8\mathbb{Z}$ is a cyclic group of order $8$ under addition, and $(\mathbb{Z}/15\mathbb{Z})^{\times}$ is not cyclic, they are not isomorphic.
\subsection{} % 1b
Both group are cyclic groups with order 4, so they are isomorphic. Define $u_4=\{z\in \mathbb{C}\backslash\{0\}:z^4=1\}=\{-1, 1, -i, i\}$. Define:
$$\varphi:\mathbb{Z}/4\mathbb{Z}\rightarrow u_4,\varphi{(\bar{k})}=i^{k}$$.

Suppose $a,b\in\bar{k}$ then $(a-b)$ is a multiple of $4$, so $i^{a}=i^b$. Thus $\varphi$ is well-defined. For simplicity, I denote the member of $\bar{k}$ class as $[k]$. For any classes $a,b$, we have $\varphi([a]+[b])=\varphi([a+b])=i^{a+b}=i^a i^b=\varphi([a])\varphi([b])$, thus, this is an isomorphism.

Since $\varphi([k])=1$ if and only if $i^k=1, k\equiv0\pmod{4}$, so $\ker\varphi={[0]}$, only contains the identity element in $\mathbb Z/4\mathbb Z$, so $\varphi$ is injective, also the cardinality of the two groups are the same, so $\varphi$ must be a bijection, $\varphi$ is an isomorphism.
\subsection{} % 1c
Define:
$$\varphi:\mathbb{Z}\rightarrow 3\mathbb Z,\varphi(x)=3x,x\in\mathbb Z$$ 

Since $\varphi(x+y)=3x+3y=\varphi(x)+\varphi(y)$, it's trivially a homomorphism. Also $\ker{\varphi}=\{0\}$ where $0$ is the identity of $\mathbb Z$, also the only element mapped to $0$ (the identity in $3\mathbb Z$), thus it's injective. Also for every element $y$ in $3\mathbb Z$, we can always find an corresponding element $\frac{y}{3}$ in $\mathbb Z$ (since $y$ is divisible by $3$), so $\varphi$ is surjective. Hence, $\varphi$ is one isomorphism.

\subsection{} % 1d
The second group contains infinite elements, but all the elements have finite order, hence, it's not cyclic. Since $\mathbb Z$ with additive operation is infinitly cyclic, the two groups can't be isomorphic. Let the second set be $G$, the identity element is $1$ since all elements multiplied by it is themselves, and for any $g\in G,g^n=1,n\in\mathbb Z$, its inverse is $g^{n-1}$ since $gg^{n-1}=1,g^{n-1}g=1$. The associativity is trivially true. For any $x,y\in G$, suppose $x^n=1,y^m=1$ for some integers $n,m$, then since $(xy)^{nm}=1,nm\in\mathbb Z\Rightarrow xy\in G$, $G$ is closed under multiplication. Also, $G$ is a non-empty subset of $\mathbb{C}\backslash\{0\}$. In conlusion, $G$ is the subgroup of $\mathbb{C}\backslash\{0\}$.
\subsection{} % 1e
$D_3=\{e,s,r,r^2,sr,sr^2\}$ and $S_3=\{e,(12),(23),(13),(123),(132)\}$, we can define $\varphi:D_3\rightarrow S_3$ by:
$$\begin{cases}s\mapsto (12) \\ r\mapsto (123)\end{cases}$$

This is one isomorphism. Instead of checking all 36 pairs, we use generators and relations of $D_3$ to check the homomorphism. Since $\varphi(r^3)=\varphi(r)^3=(123)(123)(123)=e,\varphi(s^2)=\varphi(s)^2=(12)(12)=e,\varphi(srs^{-1})=\varphi(s)\varphi(r)\varphi(s^{-1})=(12)(123)(12)=(132)=(123)^{-1}$, $\varphi$ preserves the structure of $D_3$.

Since $\ker\varphi=\{e\}$ and all permutations in $S_3$ can be generated using operations (flip and rotations) in $D_3$, i.e., $\varphi{(D_3)}=S_3$. Thus, $\varphi$ is bijective. $\varphi$ is an isomorphism.


\subsection{} % 1f
Since $|S_4|=4!=24$ and $|D_4|=|\{e,s,r,r^2,r^3,rs,r^2s,r^3s\}|=8$ ($e$ is the identity), their order are different, there can't be an bijection. Hence, they are not isomorphic.
\subsection{} % 1g
$Q=\{1, -1, i, j, k, -i, -j, -k\}$ and $T=\{1,a,a^2,a^3,b,ab,a^2b,a^3b\}$

An isomorphism is defined as:
$$f:Q\rightarrow T, f(i)=a,f(j)=b$$

Let's check if $f$ a homomorphism, preserving the relations given in $T$:
\begin{align*}
f(-1)=f(i^2)=(f(i))^2=a^2=f(j^2)=(f(j))^2=b^2 \\
f(k)=f(ij)=ab \\
f(-k)=f(ji)=f(j)f(i)=ba \\
f(-k)=f(-1)f(k)=a^3b \\
ba=a^3b \\
f(i^4)=f(1)=(f(i))^4=a^4=1
\end{align*}

Thus, since $f$ preserves the relations in $T$, it's a homomorphism. Also:
\begin{align*}
f(1)=1 \\
f(-1)=a^2 \\
f(i)=a \\
f(j)=b \\
f(k)=f(ij)=ab\\
f(-i)=f(-1)f(i)=a^3\\
f(-j)=f(-1)f(j)=a^2b\\
f(-k)=f(-1)f(k)=a^3b
\end{align*}

The homomorphism $f$ is an 1 to 1 function and all elements in $T$ can be mapped from one unique element in $Q$. Hence $f$ is bijective, and $f$ is an isomorphism!

\subsection{} % 1h
They are isomorphic. Since any subgroups of a cyclic group is cyclic. Let $G=\langle x\rangle$, and nontrivial subgroup has the form $H=\langle x^m\rangle$ for some integer $m$. Define:
$$\varphi:G\rightarrow H,\varphi(g)=g^{m},g\in G$$ 
Since $\varphi(ab)=(ab)^{m}=a^mb^m=\varphi(a)\varphi(b),\forall a,b\in G$, it's a homomorphism. If $\varphi(x^k)=e, x^{km}=e$, the only possible $k$ is zero, so $\ker\varphi=e$, where $e$ is the identity element in $G$ and $H$, $\varphi$ is injective. Also every element in $H$ has the form $(x^m)^k=\varphi(x^k)$, it's also surjective. Thus, $\varphi$ is an isomorphism between $G$ and $H$.
\newpage
\section{} %2
Suppose $\big(\begin{smallmatrix} a & -b\\b & a\end{smallmatrix}\big)$ and $\big(\begin{smallmatrix} c & -d\\d & c\end{smallmatrix}\big)$ are two elements in $G$. Then we have:
\begin{align*}
\varphi{(
\begin{pmatrix}a & -b\\b & a\end{pmatrix}+\begin{pmatrix}c & -d\\d & c\end{pmatrix})} &=\begin{pmatrix}a+c & -b-d\\b+d & a+c\end{pmatrix} \\
&= (a+c)+(b+d)i \\
&= (a+bi)+(c+di) \\
&= \varphi{(\begin{pmatrix}a & -b\\b & a\end{pmatrix})}+\varphi{(\begin{pmatrix}c & -d\\d & c\end{pmatrix})}
\end{align*}

Thus, $\varphi:G\rightarrow \mathbb{C}$ is a homomorphism. To prove $\varphi$ is isomorphic, we need to prove the injectivity and surjectivity.
Suppose $\big(\begin{smallmatrix} a & -b\\b & a\end{smallmatrix}\big)$ and $\big(\begin{smallmatrix} c & -d\\d & c\end{smallmatrix}\big)$
For the injectivity, suppose $\big(\begin{smallmatrix} a & -b\\b & a\end{smallmatrix}\big)$ and $\big(\begin{smallmatrix} c & -d\\d & c\end{smallmatrix}\big)$ are distinct elements in $G$, i.e., $a\ne c\vee b\ne d$. Since $\varphi(\big(\begin{smallmatrix} c & -d\\d & c\end{smallmatrix}\big))=c+di$, $\varphi(\big(\begin{smallmatrix} a & -b\\b & a\end{smallmatrix}\big))=a+bi\ne\varphi(\big(\begin{smallmatrix} c & -d\\d & c\end{smallmatrix}\big))$, the $\varphi$ is injective.

For every element $x=r+si$ in $\mathbb{C}$, where $r,s\in\mathbb{R}$, it can correspond to a unique real matrix $\big(\begin{smallmatrix} r& -s\\s & r\end{smallmatrix}\big)$ in $G$. Thus, $\varphi(G)=\mathbb{C}$, the surjectivity is proved. Hence, $G$ and $\mathbb{C}$ are isomorphic!

\newpage
\section{} %3
\subsection{}
\noindent\rule{\textwidth}{1pt}
\textbf{Lemma 1: Any transposition $(ab),b>a$ can be written as a product of simple transpositions (which taking the form of $(i\ i+1)$, $i\in \{1,2,\dots,n-1\}$, where $n$ is the size of permutations).}

\begin{proof}
We can induct on $k=b-a$, case $k=1$ is trivial. Suppose that any transposition $(ab)$ can be written as the product of simple transposition when $b-a=k$. When $b-a=k+1$, we have:
$$(ab)=(a\ a+1)(a+1\ b)(a\ a+1)$$
Since $b-(a+1)=k$ and $(a+1)-a=1$, all operations above are feasible. Hence, any transposition $(ab),b>a$ can be written as the product of simple transpositions.
\end{proof}

\noindent\textbf{Lemma 2: Any permutations on $n$ numbers can be generated using $(12)$ and $(12\dots n)$.}

\begin{proof}
Let $\sigma=(12\dots n)$, note that $\sigma(12)\sigma^{-1}=(23)$. By reasoning inductively, we will find $\sigma^k (12)\sigma^{-k}=(k+1\ k+2)$, hence we obtain all simple transpositions. By lemma 1, we can generate all permutations using simple permutations, thus, the lemma is proved. 
\end{proof}



\noindent\textbf{Algorithm: Bubble Sort}
\begin{Verbatim}[numbers=left]
Function bubbleSort(Type data[1..n])
    Index i, j;
    For i from n to 2 do
        For j from 1 to i - 1 do
            If data[j] > data[j + 1] then
                Exchange data[j] and data[j + 1]
End
\end{Verbatim}

The algorithm works correctly, since in each outer loop we moves $n$th smallest to the $n$th position. In the end, the array is sorted in non-decreasing order.

\noindent\rule{\textwidth}{1pt}

Suppose $H=\langle h\rangle,h\in G$ is a normal subgroup of $G$. Since any subgroup of a cyclic group is cyclic, the subgroup $N$ of $H$ can be written as the form $\langle h^d\rangle,d\in \mathbb{Z}$.

Suppose $g\in G$, since $H$ is normal in $G$, $g^{-1}hg=h^{i}\in H$ for some integer $i$ in $[1,n]$ Then for any integer $r$, we have:
$$g^{-1}(h^{d})^rg=(g^{-1}hg)^{rd}=(h^d)^{ir}\in N$$

Thus, for all $g\in G$ and elements in $N=\langle x^d\rangle$ (Let $n\in N$), we have $g^{-1}ng\in N$, $N$ is normal in $G$. 
\subsection{}

Let $G=S_4$, the symmetry group on 4 letters. Let $H=V=\{e,(12)(34),(14)(23),(13)(24)\}$, the Klein four group. Obviously $H$ is a non-empty subset of $G$. Since every non-identity element in $H$ is a product of two disjoint transposition, their order is $2$, i.e, their inverses are themselves, so also in $H$. We need to show that for any two elements $x,y$ in $H$, $xy^{-1}=xy\in H$. By some calculating, we can find that the product of any element and itself is the identity, and the product of any two distinct elements is also a double transposition (two disjoint transposition), thus in $H$. By subgroup criterion, $H\le G$.

By lemma 1,2 and the algorithm of bubble sort, by doing all the steps by bubble sort reversely (on an array consists of $n$ distinct element from $1$ to $n$), we know all permutations are feasible using only swaps, also all the swaps between adjacent elements are feasible by lemma 1, we know $G$ can be generated using $u=(12)$ and $v=(1234)$ by lemma 2. 

To show $H$ is normal in $G$, checking $uHu^{-1}=H$ and $vHv^{-1}=H$ is enough, since any elements can be expressed as the product of $u$ and $v$. And by some easy calculations we know $uHu^{-1}=H$ and $vHv^{-1}=H$ are both true. Hence, $H$ is normal in $G$.

Let $N=\{e, (12)(34)\}$, since $hNh^{-1}=N\forall h\in H$, $N$ is a normal subgroup of $H$ ($N$ is non-empty, it's closed under taking inverse and product/compositions, so $N\le H$). Taking $g=(123)\in G$, since $gNg^{-1}=\{e,(14)(23)\}\ne N$, $N$ is not normal in $H$. This is a counterexample.
\newpage
\section{} %4

\noindent\rule{\textwidth}{1pt}

\textbf{Claim: } Left cosets are in bijection via left multiplication. In other words, given a group $G$, a subgroup $H$, and two left cosets $xH, yH$ of $H$, where $x,y\in G$, left multiplication by $yx^{-1}$ creates a bijection between $xH$ and $yH$.

\begin{proof}
We can prove the correctness, injectivity, surjectivity of the mapping. Suppose $x,y$ are in $G$.

First note that if $g=xh,\ h\in H,\ x\in G$ then $(yx^{-1})g=yh$, thus the left multiplication of $yx^{-1}$ can map any elements in $xH$ to $yH$. Here proves the correctness.

Given two distinct elements $xh_1, xh_2\in xH,\ h_1,h_2\in H$, $(yx^{-1})xh_1=yh_1$ and $(yx^{-1})xh_2=yh_2$ are also distinct since if $yh_1=yh_2$ then we will get $xh_1=xh_2$ (by cancelling $yx^{-1}$), contradiction appeared. Thus, the map is injective.

Every element in $yH$ takes the form as $yh=(yx^{-1})xh,h\in H$, it arises as the image of left multiplication by $yx^{-1}$. Thus, the map is surjective. From these properties, we know left cosets are in bijection via left multiplication.
\end{proof}

\noindent\rule{\textwidth}{1pt}

Take any $g\in G$ and $n\in N$. Since $\varphi$ is a homomorphism and $H$ is abelian, we have:
\begin{align*}
    \varphi(gng^{-1}n^{-1}) &=\varphi(g)\varphi(n)\varphi(g^{-1})\varphi(n^{-1}) \\
    &= \varphi(g)\varphi(g^{-1})\varphi(n)\varphi(n^{-1}) \\
    &= \varphi(gg^{-1})\varphi(nn^{-1})=e_{H}
\end{align*}
Where $e_{H}$ is the identity element of $H$, thus, $gng^{-1}n^{-1}\in\mathrm{ker\varphi}$. By hypothesis, $\mathrm{ker\varphi}\in N$, so $gng^{-1}n^{-1}\in N$. Since $gng^{-1}=(gng^{-1}n^{-1})n\in N$, we can conclude that $gNg^{-1}\subset N\ \forall g\in G$. By the claim above, we can easily know the size of left coset of subgroup $N$ is the same as $N$, so is the right coset (The bijectivity is also proved in the same way as claim.). Thus, $|gNg^{-1}|=N$, which implies $gNg^{-1}=N$, $N$ is the normal subgroup of $G$.

\end{document}
