\documentclass[12pt]{article}
\usepackage[margin=2cm, a4paper]{geometry}
\usepackage{setspace}
\usepackage{float}
\usepackage{amsmath}
\usepackage{fancyvrb}
\usepackage{amssymb}
\usepackage{enumitem}

\usepackage{amsthm}
\usepackage[english]{babel}
\newtheorem{theorem}{Theorem}

\usepackage{xeCJK}
\setCJKmainfont{Noto Serif CJK TC}

\newcommand{\mytitle}{Abstract algebra I Homework 3 uwu}
\newcommand{\myauthor}{B13902022 賴昱錡}

% to fit the hw format
\renewcommand{\thesection}{\arabic{section})}
\renewcommand{\thesubsection}{(\alph{subsection})}

\usepackage{fancyhdr}
\pagestyle{fancy}
\fancyhead{}
\fancyhead[L]{\mytitle}
\fancyhead[R]{\myauthor}

\title{\textbf{\mytitle}}
\author{\textbf{\myauthor}}
\date{Due: 1st October 2025}

\begin{document}

\onehalfspacing
\maketitle

\section{} %1
\subsection{} % 1a
\noindent\rule{\textwidth}{1pt}

\textbf{Claim: }If $G$ is cyclic and $H$ is isomorphic to $G$, then $H$ is cylic.
\begin{proof}
Suppose the isomorphism is $\varphi: G\rightarrow H$ and $G=\langle x\rangle\infty$, then every element of $H$ can be written as the form $\varphi(x^d)$ for some integer $d$, also: $\varphi(x^d)=\varphi(x)^d$. Hence, $H$ is generated by $\varphi(x)$, i.e., by claim above, $H=\langle \varphi(x)\rangle$ is cyclic.
\end{proof}

\noindent\rule{\textwidth}{1pt}


$(\mathbb{Z}/15\mathbb{Z})^{\times}=\{\bar{1},\bar{2},\bar{4},\bar{7},\bar{8},\bar{11},\bar{13},\bar{14}\}$ and $\mathbb{Z}/8\mathbb{Z}=\{\bar{0},\bar{1},\bar{2},\bar{3},\bar{4},\bar{5},\bar{6},\bar{7}\}$. Since $\mathbb{Z}/8\mathbb{Z}$ is a cyclic group of order $8$ under addition, and $(\mathbb{Z}/15\mathbb{Z})^{\times}$ is not cyclic, they are not isomorphic.
\subsection{} % 1b
\subsection{} % 1c
\subsection{} % 1d
\subsection{} % 1e
\subsection{} % 1f
\subsection{} % 1g
\subsection{} % 1h

\newpage
\section{} %2
Suppose $\big(\begin{smallmatrix} a & -b\\b & a\end{smallmatrix}\big)$ and $\big(\begin{smallmatrix} c & -d\\d & c\end{smallmatrix}\big)$ are two elements in $G$. Then we have:
\begin{align*}
\varphi{(
\begin{pmatrix}a & -b\\b & a\end{pmatrix}+\begin{pmatrix}c & -d\\d & c\end{pmatrix})} &=\begin{pmatrix}a+c & -b-d\\b+d & a+c\end{pmatrix} \\
&= (a+c)+(b+d)i \\
&= (a+bi)+(c+di) \\
&= \varphi{(\begin{pmatrix}a & -b\\b & a\end{pmatrix})}+\varphi{(\begin{pmatrix}c & -d\\d & c\end{pmatrix})}
\end{align*}

Thus, $\varphi:G\rightarrow \mathbb{C}$ is a homomorphism. To prove $\varphi$ is isomorphic, we need to prove the injectivity and surjectivity.
Suppose $\big(\begin{smallmatrix} a & -b\\b & a\end{smallmatrix}\big)$ and $\big(\begin{smallmatrix} c & -d\\d & c\end{smallmatrix}\big)$
For the injectivity, suppose $\big(\begin{smallmatrix} a & -b\\b & a\end{smallmatrix}\big)$ and $\big(\begin{smallmatrix} c & -d\\d & c\end{smallmatrix}\big)$ are distinct elements in $G$, i.e., $a\ne c\vee b\ne d$. Since $\varphi(\big(\begin{smallmatrix} c & -d\\d & c\end{smallmatrix}\big))=c+di$, $\varphi(\big(\begin{smallmatrix} a & -b\\b & a\end{smallmatrix}\big))=a+bi\ne\varphi(\big(\begin{smallmatrix} c & -d\\d & c\end{smallmatrix}\big))$, the $\varphi$ is injective.

For every element $x=r+si$ in $\mathbb{C}$, where $r,s\in\mathbb{R}$, it can correspond to a unique real matrix $\big(\begin{smallmatrix} r& -s\\s & r\end{smallmatrix}\big)$ in $G$. Thus, $\varphi(G)=\mathbb{C}$, the surjectivity is proved. Hence, $G$ and $\mathbb{C}$ are isomorphic!

\newpage
\section{} %3
\subsection{}
Suppose $H=\langle h\rangle,h\in G$ is a normal subgroup of $G$. Since any subgroup of a cyclic group is cyclic, the subgroup $N$ of $H$ can be written as the form $\langle h^d\rangle,d\in \mathbb{Z}$.

Suppose $g\in G$, since $H$ is normal in $G$, $g^{-1}hg=h^{i}\in H$ for some integer $i$ in $[1,n]$ Then for any integer $r$, we have:
$$g^{-1}(h^{d})^rg=(g^{-1}hg)^{rd}=(h^d)^{ir}\in N$$

Thus, for all $g\in G$ and elements in $N=\langle x^d\rangle$ (Let $n\in N$), we have $g^{-1}ng\in N$, $N$ is normal in $G$. 
\subsection{}
Let $G=S_4$, the symmetry group on 4 letters. Let $H=V=\{e,(12)(34),(14)(23),(13)(24)\}$, the Klein four group. Obviously $H$ is a non-empty subset of $G$. Since every non-identity element in $H$ is a product of two disjoint transposition, their order is $2$, i.e, their inverses are themselves, so also in $H$. We need to show that for any two elements $x,y$ in $H$, $xy^{-1}=xy\in H$. By some calculating, we can find that the product of any element and itself is the identity, and the product of any two distinct elements is also a double transposition (two disjoint transposition), thus in $H$. By subgroup criterion, $H\le G$.

Since $G$ can be generated using $u=(12)$ and $v=(1234)$ by definition. To show $H$ is normal in $G$, checking $uHu^{-1}=H$ and $vHv^{-1}=H$ is enough, since any elements can be expressed as the product of $u$ and $v$. And by some easy calculations we know $uHu^{-1}=H$ and $vHv^{-1}=H$ are both true. Hence, $H$ is normal in $G$.

Let $N=\{e, (12)(34)\}$, since $hNh^{-1}=N\forall h\in H$, $N$ is a normal subgroup of $H$ ($N$ is non-empty, it's closed under taking inverse and product/compositions, so $N\le H$). Taking $g=(123)\in G$, since $gNg^{-1}=\{e,(13)(24)\}\ne N$, $N$ is not normal in $H$. This is a counterexample.
\newpage
\section{} %4

\noindent\rule{\textwidth}{1pt}

\textbf{Claim: } Left cosets are in bijection via left multiplication. In other words, given a group $G$, a subgroup $H$, and two left cosets $xH, yH$ of $H$, where $x,y\in G$, left multiplication by $yx^{-1}$ creates a bijection between $xH$ and $yH$.

\begin{proof}
We can prove the correctness, injectivity, surjectivity of the mapping. Suppose $x,y$ are in $G$.

First note that if $g=xh,\ h\in H,\ x\in G$ then $(yx^{-1})g=yh$, thus the left multiplication of $yx^{-1}$ can map any elements in $xH$ to $yH$. Here proves the correctness.

Given two distinct elements $xh_1, xh_2\in xH,\ h_1,h_2\in H$, $(yx^{-1})xh_1=yh_1$ and $(yx^{-1})xh_2=yh_2$ are also distinct since if $yh_1=yh_2$ then we will get $xh_1=xh_2$ (by cancelling $yx^{-1}$), contradiction appeared. Thus, the map is injective.

Every element in $yH$ takes the form as $yh=(yx^{-1})xh,h\in H$, it arises as the image of left multiplication by $yx^{-1}$. Thus, the map is surjective. From these properties, we know left cosets are in bijection via left multiplication.
\end{proof}

\noindent\rule{\textwidth}{1pt}

Take any $g\in G$ and $n\in N$. Since $\varphi$ is a homomorphism and $H$ is abelian, we have:
\begin{align*}
    \varphi(gng^{-1}n^{-1}) &=\varphi(g)\varphi(n)\varphi(g^{-1})\varphi(n^{-1}) \\
    &= \varphi(g)\varphi(g^{-1})\varphi(n)\varphi(n^{-1}) \\
    &= \varphi(gg^{-1})\varphi(nn^{-1})=e_{H}
\end{align*}
Where $e_{H}$ is the identity element of $H$, thus, $gng^{-1}n^{-1}\in\mathrm{ker\varphi}$. By hypothesis, $\mathrm{ker\varphi}\in N$, so $gng^{-1}n^{-1}\in N$. Since $gng^{-1}=(gng^{-1}n^{-1})n\in N$, we can conclude that $gNg^{-1}\subset N\ \forall g\in G$. Since the left coset, right coset and the conjugate have the same size with subgroup, i.e. $|gNg^{-1}|=N$, $gNg^{-1}=N\ \forall g\in G$ is true, and $N$ is normal subgroup of $G$. By the claim above, we can easily know the size of left coset of subgroup $N$ is the same as $N$, so is the right coset (The bijectivity is also proved in the same way as claim.). Thus, $|gNg^{-1}|=N$, which implies $gNg^{-1}=N$, $N$ is the normal subgroup of $G$.

\end{document}
